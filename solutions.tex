%%%%%%%%%%%%%%%%%%%%%%%%%%%%%%%%%%%%%%%%%%%%%%%
%%%This is a science homework template. Modify the preamble to suit your needs. 
%The junk text is   there for you to immediately see how the headers/footers look at first 
%typesetting.


\documentclass[12pt]{article}

\usepackage[utf8]{inputenc} 

% packages
\usepackage{amsmath} 
%geometry (sets margin) and other useful packages
\usepackage[margin=1.25in]{geometry}
\usepackage{graphicx,ctable,booktabs}
\usepackage[obeyspaces]{url}% http://ctan.org/pkg/url
\usepackage{hyperref}
\usepackage{enumitem}
\usepackage{lmodern} % for bold teletype font
\usepackage{minted}




%
%Redefining sections as problems
%
\makeatletter
\newenvironment{problem}{\@startsection
	{section}
	{1}
	{-.2em}
	{-3.5ex plus -1ex minus -.2ex}
	{2.3ex plus .2ex}
	{\pagebreak[3]%forces pagebreak when space is small; use \eject for better results
		\large\bf\noindent{Exercise }
	}
}
{\vspace{0.8cm}}
%{\vspace{1ex}\begin{center} \rule{0.3\linewidth}{.3pt}\end{center}}
%	\begin{center}\large\bf \ldots\ldots\ldots\end{center}}
\makeatother


%
%Fancy-header package to modify header/page numbering 
%
\usepackage{fancyhdr}
\pagestyle{fancy}
%\addtolength{\headwidth}{\marginparsep} %these change header-rule width
%\addtolength{\headwidth}{\marginparwidth}
\lhead{Exercise \thesection}
\chead{} 
\rhead{\thepage} 
\lfoot{\small\scshape Resources of Open Data} 
\cfoot{} 
\rfoot{\footnotesize DH Estonia} 
\renewcommand{\headrulewidth}{.3pt} 
\renewcommand{\footrulewidth}{.3pt}
\setlength\voffset{-0.25in}
\setlength\textheight{648pt}



%%%%%%%%%%%%%%%%%%%%%%%%%%%%%%%%%%%%%%%%%%%%%%%

%
%Contents of problem set
%    
\begin{document}
	
	\title{Resources of Open Data: Exercise Solutions}
	\author{Maksim Misin}
	\date{November 3, 2017}
	
	\maketitle
	
	\thispagestyle{empty}
	
	%Example problems
	\begin{problem}{\it OpenAIRE}

	
	\begin{enumerate}[label=\textbf{\alph*)},leftmargin=*]
		\item Root URL: \url{http://api.openaire.eu/search}. Endpoints are: \texttt{publications}, \texttt{datasets}, \texttt{projects}.
		
		By default response is in XML format. To obtain JSON response add \texttt{format=json} parameter to URL.
		
		\item The number of returned datasets is specified in \texttt{response/header/query/size} part of the response. \texttt{response/header/query/total} shows the total number of results satisfying search criteria. To narrow the results add \texttt{size=300} parameter to query.
		
		\item To get results in CSV format use \texttt{format=csv}.
		
	\end{enumerate}
		
	\end{problem}
	
	
	\begin{problem}{\it Figshare}
		
	\begin{enumerate}[label=\textbf{\alph*)},leftmargin=*]
	\item 15 most viewed articles: 
	\begin{minted}[tabsize=4]{text}
https://api.figshare.com/v2/articles?
	page_size=15&
	order=views&
	order_direction=desc
	\end{minted}
	and collections:
	\begin{minted}[tabsize=4]{text}
https://api.figshare.com/v2/collections?
	page_size=15&
	order=views&
	order_direction=desc
\end{minted}

	
	\item For example, to search for "social media" datasets we send a POST request to \url{https://api.figshare.com/v2/articles/search} endpoint with the following body:
	\begin{minted}[tabsize=4]{js}
{
	"item_type": 3,
	"search_for": "social media",
	"page": 1,
	"page_size": 1000,
	"order": "views",
	"order_direction": "desc",
	"published_since": "2016-01-01"
}
	\end{minted}
	
	\item An example of article details endpoint: \url{https://api.figshare.com/v2/articles/5539834}. Download links are associated with \texttt{download\_url} keys. To download all files you can for example use python \texttt{urllib} and \texttt{json} libraries.
	
	
	
	\end{enumerate}
	
		
	\end{problem}


	\begin{problem}{\it Plos ONE}
	
	
	\begin{enumerate}[label=\textbf{\alph*)},leftmargin=*]
		\item 	\begin{minted}[tabsize=4]{text}
http://api.plos.org/search?
	q=abstract:economic AND body:Estonia&
	wt=json
		\end{minted}	
		
		\item  \begin{minted}[tabsize=4]{text}
http://api.plos.org/search?
	q=subject:"social sciences" AND
		publication_date:[2014-01-01T00:00:00Z TO 2015-01-01T00:00:00Z] AND 
		doc_type:full&
	wt=json&
	fl=title,pagecount,counter_total_all,alm_scopusCiteCount&
	rows=500
		\end{minted}	
		
		\item Assuming that we got the following response:
		\begin{minted}[tabsize=4]{js}
{
	"response": {
		"numFound": 4521,
		"start": 0,
		"docs": [
			{
				"alm_scopusCiteCount": 6,
				"counter_total_all": 47605,
				"pagecount": 7,
				"title": "Globalization and Econ..."
			},
			{
				"alm_scopusCiteCount": 18,
				"counter_total_all": 24113,
				"pagecount": 15,
				"title": "Gender on the Brain: A Cas..."
			},
			{
				"alm_scopusCiteCount": 2,
				"counter_total_all": 4961,
				"pagecount": 8,
				"title": "Evidence for the Identifi..."
			}
		]
	}
}
\end{minted}


	we delete everything unrelated to the articles:
		\begin{minted}[tabsize=4]{js}
[
	{
		"alm_scopusCiteCount": 6,
		"counter_total_all": 47605,
		"pagecount": 7,
		"title": "Globalization and Econ..."
	},
	{
		"alm_scopusCiteCount": 18,
		"counter_total_all": 24113,
		"pagecount": 15,
		"title": "Gender on the Brain: A Cas..."
	},
	{
		"alm_scopusCiteCount": 2,
		"counter_total_all": 4961,
		"pagecount": 8,
		"title": "Evidence for the Identifi..."
	}
]
\end{minted}
	\end{enumerate}

	and after conversion with \url{https://sqlify.io} obtain the following CSV:
\begin{minted}[tabsize=4]{text}
alm_scopusCiteCount,counter_total_all,pagecount,title
6,47605,7,Globalization and Econ...
18,24113,15,Gender on the Brain: A Cas...
2,4961,8,Evidence for the Identifi...
\end{minted}

	The subsequent analyses can be done in a number of ways depending on your method of choice for tabulated data manipulations.

\end{problem}

\begin{problem}{\it Figshare}
	
	\begin{enumerate}[label=\textbf{\alph*)},leftmargin=*]
\item Assuming we picked Swiss medieval manuscripts repository, the formatted manifest file is going to look like this:
\begin{minted}[tabsize=4]{js}
{
	"@context": "http://iiif.io/api/presentation/2/context.json",
	"@id": "http://www.e-codices.unifr.ch/metadata/iiif/kba-0...",
	"@type": "sc:Manifest",
	"label": "Aarau, Aargauer Kantonsbibliothek, MsMurF 3",
	"metadata": [
		{
			"label": "Location",
			"value": "Aarau"
		},
		{
			"label": "Date",
			"value": [
				{
					"@value": "1508",
					"@language": "de"
				},
				{
					"@value": "1508",
					"@language": "en"
				},
				{
					"@value": "1508",
					"@language": "fr"
				},
				{
					"@value": "1508",
					"@language": "it"
				}
			]
		}
	],
	"description": [
		{
			"@value": "Pontificale für Johannes Feie...",
			"@language": "de"
		...
}

\end{minted}

	\item The first image URL in case of the above manuscript is the following: 
	\begin{minted}{text}
http://www.e-codices.unifr.ch/loris/kba/kba-0003/
	kba-0003_e001.jp2/full/full/0/default.jpg
	\end{minted}
	
	\item The JSONPath returning links to images for the above manifest is:
	\mint{text}|$.sequences[*].canvases[*].images[*].resource.@id|
	
	After copying links and removing all commas and quotes the resulting file should look like this:
	
	\begin{minted}{text}
http://www.e-codices.unifr.ch:80/l...03__e001.jp2/full/full/0/default.jpg
http://www.e-codices.unifr.ch:80/l...03__e005.jp2/full/full/0/default.jpg
http://www.e-codices.unifr.ch:80/l...03__000a.jp2/full/full/0/default.jpg
http://www.e-codices.unifr.ch:80/l...03__000b.jp2/full/full/0/default.jpg
http://www.e-codices.unifr.ch:80/l...03__000c.jp2/full/full/0/default.jpg
http://www.e-codices.unifr.ch:80/l...03__000d.jp2/full/full/0/default.jpg
...
	\end{minted}
	Then we can download all links with \texttt{wget} via 
	\mint{bash}|wget -i links.txt|
	\end{enumerate}
\end{problem}
	
\end{document}
